\documentclass[border=15pt, multi, tikz]{standalone}

% 加载所需的包和库
\usepackage{import}
\usetikzlibrary{positioning, calc, decorations.markings}

% 假设您的 init.tex 位于上一级的 layers 文件夹中
\subimport{../layers/}{init} 

% --- 颜色定义 ---
\definecolor{InputColor}{RGB}{200,200,200}
\definecolor{ConvColor}{RGB}{156,201,229}
\definecolor{ConvReluColor}{RGB}{179,205,224}
\definecolor{TSMColor}{RGB}{255,228,181}
\definecolor{SumColor}{RGB}{150,223,150}
\definecolor{DWInvColor}{RGB}{178,223,219} % 为DepthWise Inverse模块定义新颜色 (青色)

\pgfdeclarelayer{background}
\pgfdeclarelayer{main}
\pgfsetlayers{background,main}


\begin{document}

\begin{tikzpicture}[
    block/.style={font=\Large},
    connection/.style={
        ultra thick, 
        draw=black, 
        opacity=0.6,
        -latex
    },
    midarrow/.style={
        decoration={
            markings,
            mark=at position 0.5 with {\arrow{latex}}
        },
        postaction={decorate}
    }
]

% ====================================================================================
%                     全局变量定义
% ====================================================================================
\def\zshift{4.0}  % 稍微增加Z轴偏移,让长分支更清晰
\def\xposA{0.0}   % 第1列位置
\def\xposB{3}   % 第2列位置
\def\xposC{6}   % 第3列位置
\def\xposD{9}  % 第4列位置 (两个分支的对齐终点)
\def\xposE{12}  % Sum模块位置


% ====================================================================================
%                            Stage 1: Module Definitions
% ====================================================================================
\begin{pgfonlayer}{main}
% --- 分支2: 底部 (2个模块, 与顶部对齐) ---
\pic at (\xposA, 0, -\zshift) {RightBandedBox={name=conv3x3_1, fill=ConvColor, bandfill=ConvReluColor, height=22, width={4,0}, depth=22, xlabel={{"3$\times$3$\times$3 Conv",""}}, ylabel=}};
\pic at (\xposD, 0, -\zshift) {RightBandedBox={name=conv3x3_2, fill=ConvColor, bandfill=ConvReluColor, height=22, width={4,0}, depth=22, xlabel={{"3$\times$3$\times$3 Conv",""}}, ylabel=}};

% --- 分支1: 顶部 (4个模块) ---
\pic at (\xposA, 0, \zshift) {RightBandedBox={name=conv1x1, fill=ConvColor, bandfill=ConvReluColor, height=22, width={4,0}, depth=22, xlabel={{"1$\times$1$\times$1 Conv",""}}, ylabel=}};
\pic at (\xposB, 0, \zshift) {Box={name=dwinv1, fill=DWInvColor, height=22, width=4, depth=22, xlabel={{"DepthWise Inverse",""}}, ylabel=}};
\pic at (\xposC, 0, \zshift) {Box={name=tsm, fill=TSMColor, height=22, width=4, depth=22, xlabel={{"TSM",""}}, ylabel=}};
\pic at (\xposD, 0, \zshift) {Box={name=dwinv2, fill=DWInvColor, height=22, width=4, depth=22, xlabel={{"DepthWise Inverse",""}}, ylabel=}};


% --- Sum模块 ---
\pic at (\xposE, 0, 0) {RightBandedBox={name=sum, fill=SumColor, height=25, width={4}, depth=25, xlabel={{"Sum",""}}, ylabel=}};
\end{pgfonlayer}

% ====================================================================================
%                Stage 2: Connections
% ====================================================================================
\begin{pgfonlayer}{background}

% --- Input and Split Point ---
\node (input_label) at (-5, 0, 0) [font=\Large] {Input};
\coordinate (input_point) at (-4, 0, 0);
\coordinate (split_point) at (-2, 0, 0);
\draw[connection, midarrow] (input_point) -- (split_point);

% --- Split to Layer 1 ---
\coordinate (turn_point_upper_1) at (-2, 0, \zshift);
\coordinate (turn_point_lower_1) at (-2, 0, -\zshift);
\draw[connection, midarrow] (split_point) -- (turn_point_upper_1) -- (conv1x1-anchor);
\draw[connection, midarrow] (split_point) -- (turn_point_lower_1) -- (conv3x3_1-anchor);

% --- Inter-Layer Connections ---
% Top Branch
\draw[connection, midarrow] (conv1x1-anchor) -- (dwinv1-anchor);
\draw[connection, midarrow] (dwinv1-anchor) -- (tsm-anchor);
\draw[connection, midarrow] (tsm-anchor) -- (dwinv2-anchor);
% Bottom Branch
\draw[connection, midarrow] (conv3x3_1-anchor) -- (conv3x3_2-anchor);

% --- Layer 2 to Merge Point ---
\coordinate (merge_point) at (\xposE-2, 0, 0);
\coordinate (turn_point_upper_2) at (\xposE-2, 0, \zshift);
\coordinate (turn_point_lower_2) at (\xposE-2, 0, -\zshift);

% 连接来自两个分支的最终输出 (dwinv2 和 conv3x3_2)
\draw[connection, midarrow] (dwinv2-anchor) -- (turn_point_upper_2) -- (merge_point);
\draw[connection, midarrow] (conv3x3_2-anchor) -- (turn_point_lower_2) -- (merge_point);
\draw[connection, midarrow] (merge_point) -- (sum-anchor);

% --- Output Point ---
\coordinate (output_point) at ($(sum-east)+(2,0,0)$);
\node (output_label) at ($(output_point)+(1,0,0)$) [font=\Large] {Output};
\draw[connection, midarrow] (sum-anchor) -- (output_point);

\end{pgfonlayer}
\node at (current bounding box.south) [below=10pt, font=\Huge\bfseries\itshape] {BFCM};
\end{tikzpicture}
\end{document}